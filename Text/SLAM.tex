
\chapter{Příznaky v digitalizovaném obraze a jejich využití}
\label{chap:slam}

V následujících kapitolách budou představeny především metody pro detekci a popis bodových příznaků. Tyto algoritmy se snaží v obraze nalézt body výrazné vzhledem k jejích okolí a popsat je tak, aby je při jejich opětovném nalezení v jiném obraze bylo možné identifikovat bez ohledu na transformaci mezi těmito obrazy (natočení, posunutí, roztažení, změna nasvětlení, ... ). %Rozšířením tohoto přístupu jsou metody SIFT, SURF a metoda maximálně stabilních extremálních oblastí (dále MSER), které namísto bodů hledají celé oblasti, které budou co nejméně podléhat vlivu obrazových transformací.

\noindent{Práce s bodovými příznaky zahrnuje:}

\begin{itemize}
	\item{detekci příznaků} - Nalezení nejvhodnějších příznakových bodů.
	\item{deskripci příznaků} - Popis příznaků tak, aby tyto byly správně identifikovány při opětovném nalezení za minimalizace vlivů osvětlení, prostorových transformací atd.
	\item{způsob porovnání deskriptorů} - Metodu, kterou budeme určovat, které dva deskriptory z různých obrazů popisují stejný bod nebo stejnou oblast.
\end{itemize} 

Některé z prezentovaných metod práce s bodovými příznaky řeší všechny tři tyto problémy, jiné jenom některé z nich a pro jejich využití je tedy potřeba zbylé doplnit. Dále jsou uvedeny dvě metody pro detekci objektů, tj. Haarovská kaskáda a histogram orientovaných gradientů, což jsou učící se algoritmy, které v obraze hledají obecně objekty určitého charakteru (typicky například obličeje nebo postavy). Nakonec jsou ještě uvedeny metody porovnání popisů bodů a odhadu prostorové transformace pomocí množiny bodů identifikovaných mezi dvěma obrazy. 

Jednou z možností využití bodových příznaků v digitalizovaném obraze je identifikace objektů v něm, která může být nasazena v bezpečnostních aplikacích, v případech, kdy je potřeba aby ovládací rozhraní systému identifikovalo uživatele, nebo například pro vyhledávání v databázi neoznačených fotografií a jejich přiřazování k sobě. Další možností je aplikace ve sledování objektů v obraze za účelem extrakce jejich pohybu, jejich počítání atd., rekonstrukce tvaru a charakteru prostředí, například za účelem pohybu a orientace v něm a lokalizace pozorovatele za tímž účelem. 

V této práci jsou metody extrakce příznaků uvažovány zejmena v kontextu využití v algoritmu simultánní lokalizace a mapování (SLAM). Jedná se úlohu vytvoření mapy prostředí a zároveň určení pozice pozorovatele v tomto prostředí, přičemž sloučení těchto úloh do jedné je klíčem k jejich řešení. Metody extrakce příznaků jsou posuzovány v kontextu technické realizace tohoto algoritmu za využití jedné nebo více kamer snímajících prostředí a systému pracujícího v reálném čase. V tomto systému jsou v každém kroku algoritmu porovnány body nalezené v aktuálním obrazu s body nalezenými dříve a odhadnuta prostorová transformace (změna polohy pozorovatele), ke které muselo dojít mezi předchozím a aktuálním obrazem.

%
%Existuje mnoho různých algoritmů k řešení tohoto problému, které se primárně liší předpokládaným technickým vybavením a prostředím jejich nasazení, zejména senzory pro získávání informace o prostředí (od 3D scannerů až po  dotykové senzory), a také použitým statistickým aparátem, který zahrnuje Kalmanův filtr, částicový filtr a další.
%
%V dalším je uvažován SLAM algoritmus, který:
%
%\begin{enumerate}
%	\item Pracuje v reálném čase (vzorkovací frekvence cca 30Hz)
%	\item Jako vstup používá jednu standartní kameru (např. webkameru)
%%	\item Jako statistický aparát používá Kalmanův filtr.
%	\item K mapování využívá příznaky
%	\item Je navržen pro fungování obecně v jakémkoli prostředí bez jeho apriorní znalosti.
%\end{enumerate}
%
%ad 4) Ačkoli existují velmi slibné metody bezpříznakové rekonstrukce prostředí v reálném čase (zejmena LSD SLAM), jejich složitost (hlavně implementační) je mimo možnosti realizace v rámci BP. Tyto metody jsou navíc z logiky věci výpočetně náročnější než hledání několika málo příznakových bodů v každém obraze, což ponechává příznakovému SLAMu prostor v oblasti aplikací na méně výkonném hardwaru.
%
%\section{Matematické Definice SLAMu}
%%\label{sec:orgheadline2}
%
%V definici SLAMu vystupují nasledující tři proměnné:
%
%\begin{description}
%	\item[{poloha pozorovatele \(x_{k}\),}] kde index \(k\) značí diskrétní čas
%	\item[{poloha orientačních bodů \(m_{k}\)}] a
%	\item[{pozorování $o_{k}$}] kde některé z algoritmů používají v každém kroku celou jeho historii \(o_{0:k}\), zatímco jiné za účelem zjednodušení úlohy a snížení její výpočetní náročnosti využijí pouze aktuální údaje, nebo nějakou jejich aproximaci na základě aktuálních a doposud vypočtených.
%\end{description}
%
%Cílem algoritmu je nalézt pravděpodobnostní rozložení
%\begin{align}
%P(m_{k},x_{k}|o_{k}),
%\end{align}
%
%kde střední hodnoty odhadovaných veličin s časem konvergují k jejich skutečným polohám a jejich kovarianční matice k nule.
%
%Při znalosti (nebo dostupnosti odhadu) rozložení \(P(x_{k}|x_{k-1})\) lze s
%pomocí Bayesova pravidla tento vztah rozložit na:
%
%\begin{align}
%P(x_{k}|o_{k}, m{k}) = P(o_{k}| x_{k}, m_{k})P(x_{k}|x_{k-1})
%P(x_{k-1}|m_{k},o_{k}) \frac{1}{Z} \\
%P(m_{k},o_{k}, x_{k}) = P(m_{k}|x_{k},m_{k-1},o_{k})
%P(m_{k-1},x_{k}|o_{k},m_{k-1}),
%\end{align}
%
%Přičemž rozložení na pravé straně rovnic lze získat z Kalmanova filtru a celkovou mapu potom střídavým obnovováním levých stran pomocí maximalizace očekávání.
%
%\section{Úloha detekce příznaků}
%
%K získání pozorování použitelného v rovnicích v předchozím bodě je potřeba mít:
%
%\begin{description}
%	\item[detektor příznaků], který vhodné příznaky v obrazu nalezne
%	\item[deskriptor příznaků], který nalezené příznaky popíše pro jejich uchování a porovnávání a
%	\item[způsob porovnání deskriptorů]
%\end{description} 
%
%Některé z prezentovaných metod řeší všechny tři tyto problémy, jiné jenom některé z nich a pro jejich zapojení do systému je tedy potřeba ty zbylé nějak doplnit.

%Také je potřeba model odhadu hloubky (respektive 3D polohy) pozorovaných příznaků, tj jejich převedení z pozorovaných příznaků na orientační body, které bude s dalšími pozorovánímí konvergovat k nule.
%
%Orientační body získané pozorováním příznaků je dále nutno umět nějakým způsobem uchovávat, a hlídat jejich množství, tj. mít metriku jejich kvality, špatné zahazovat a soustředit se na nalezení a opětovnou identifikaci těch kvalitních.
%
%Obvyklým problémem SLAMu je též tzv. loop closure, neboli uzavření smyčky. Algoritmus má totiž po čase tendenci nasčítáním malých chyb v polohách orientačních bodů vytvořit velký prostorový posun mezi vzdálenými příznaky. Při znovunavštívení již viděného místa jinou cestou (což je právě to uzavření smyčky v trajektorii pozorovatele) je potřeba místo jako celek rozpoznat a vzniklý posun v mapě opravit.

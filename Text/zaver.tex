

\chapter{Závěr}

V této práci jsou teoreticky popsány metody detekce a popisu bodových příznaků v digitalizovaném obraze. Cílem teoretické části je poskytnou čtenáři přehled těchto metod spolu s vysvětlením principu jejic fungování. Tyto metody jsou uvedeny od nejstarších a nejjednodušších, jako je Moravcův nebo Harrisův operátor, po modernější a komplexnější přístupy jako je SIFT, SURF nebo MSER. Pozornost je věnována i moderním snahám o výkonovou optimalizaci této úlohy v algoritmech FAST, BRIEF a ORB. Dále jsou zmíněny možnosti detekce objektů v obrazu pomocího se algoritmu Haar a jeho modernější alternativy metody histogramu orientovaných gradientů. V sekcích \ref{sec:line} a \ref{sec:obj} jsou zmíněny možnosti využití hran a objektů jakožto příznaků a příklady jejich nasazení v praxi. Popis metod obsahuje vždy algoritmus, vysvětlení jeho principu a případně popis využívaného matematického aparátu, jako je pyramida rozdílů gaussiánů (DoG) u metody SIFT, nebo integrální obraz a algoritmus AdaBoost u metody Haar. Teoretickou část uzavírájí algoritmy použité k další práci s nalezenými příznaky: hledání nejbližšího souseda a jedna z jeho možných aproximací Best bin first používané k přiřazování příznakových bodů. Nakonec je uvedena robustní regresní metoda RANSAC využívaná mimo jiné k odhadu vlastností geometrické transformace mezi dvěma obrazy.

Cílem praktické části práce je vybrané metody otestovat z hlediska množství nalezených příznaků, rychlosti jejich nalezení a popisu a kvality odhadu matice homografie. Jejím obsahem je popis využitého datasetu a testovaných prostorových transformací, vysvětlení pojmu homografie, algoritmus jejího hledání a výpočet měřítka, podle kterého je z ní určována kvalita jednotlivých metod. Praktickou část práce uzavírá popis softwarové implementace testovacího frameworku a okomentované výstupy z experimentů. Při souhrném testování byla zjištěna celková převaha metod SIFT a SURF nad ostatními potvrzující jejich robustností vůdči složitějším prostorovým transformacím jako je změna úhlu pozorovatele, nebo rotace.
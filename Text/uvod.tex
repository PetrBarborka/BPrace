

\chapter{Úvod}

Cílem této práce je poskytnout přehled možných přístupů k detekci a popisu příznaků v digitalizovaném obraze, ukázku jejich implementace a srovnání jejich výkonností. Práce sestává z teoretické části, kde jsou podrobně popsány použité algoritmy a části praktické, ve které je popsána implementace testovacího frameworku a prezentovány výsledky testů jednotlivých kombinací detektor - deskriptor na úloze nalezení homografie zobrazení při přechodu mezi párem testovacích obrazů.

V kapitole \ref{chap:slam} je jsou vysvětleny klíčové pojmy detekce, desripce a porovnání příznakových bodů, možné scénáře aplikací těchto algoritmů a uveden kontext, ve kterém jsou jednotlivé metody uvažovány v této práci. V kapitole \ref{chap:prehled} jsou postupně představeny metody detekce a popisu příznaků. Nejprve jsou uvedeny metody použité k nalezení a popisu bodových příznaků (Moravcův operátor až MSER), poté jsou pro doplnění zmíněny metody indetifikace objektů Haar a Histogram orientovaných gradientů, které nejsou součástí porovnání, ale jsou zajímavou alternativou při řešení některých úloh zmíněných v kapitole \ref{chap:slam}. Následují dva příklady algoritmů použitelných ke spárování deskriptorů mezi dvěma obrazy: nejbližší soused a Best Bin First. Závěrem této kapitoly je popsán robustní regresní algoritmus RANSAC, který představuje jednu z možností odhadu prostorové transformace mezi dvěma obrazy na kterých byly nalezeny a přiřazeny bodové příznaky. Kapitola \ref{chap:impl} se zabývá popisem impementace testovacího frameworku a popisuje výsledky porovnání výkonnosti párů detektor - deskriptor na použitém datsetu. Vyhodnocení výkonnosti je provedeno pomocí porovnání matice homografie zadané v datasetu a té aproximované pomocí příznakových bodů nalezených porovnávanými metodami.

